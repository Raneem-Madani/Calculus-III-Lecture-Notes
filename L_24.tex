\begin{definition}
Differentials: if $ z=f(x,y) $, then we define the differential \\
$dz=f_x \partial x + f_y \partial y $\\
$dz=\cfrac{\partial f}{\partial x} \partial x+\cfrac{\partial f} {\partial y }\partial y$\\
let:\\
$\partial x = \delta x = x-a $\\
$ \partial y = \Delta y = y-a $\\
$ \partial z =\delta z = z - z_\circ = f(x,y) - f(a,b)$\\
$ dz = \cfrac{\partial f} {\partial x}\mid_{(a,b)} (x-a) + \cfrac{\partial f}{\partial y}\mid_{(a,b)}(y-b)$
\end{definition}
\noindent{\color{smalt(darkpowderblue)}\rule{\linewidth}{.2mm}}
%-------------------------------------------------------
\begin{example}
let $t=f(x,y)=x^{2}+3xy-y^{2}$
\begin{enumerate}
    \item find the differential\\
    $\partial t = (2x + 3y) \partial x + (3x - 2y) \partial y$\\
    \item if x change from 2 to 2.05\\
    y change from 3 to 2.96 \\
    com pane $\partial z , \delta z$
\end{enumerate}
$\partial z = (2(2) + 3(3)) \cfrac{5}{100} + (3.2.2.3)\cfrac{-4}{100}$\\
$=\cfrac{65}{100} = 0.65 $\\
$(2,3)\rightarrow (2.05,2.96)$\\
$\Delta z = f((2.05,2.96)-f(2,3)) = 0.6449 $\\
$dz = \delta z = z - z_\circ$\\
\end{example}
\noindent{\color{smalt(darkpowderblue)}\rule{\linewidth}{.2mm}}
%-------------------------------------------------
functions of three variables\\
$w = f (x,y,z) $\\
$\partial w = f_x \partial x + f_y \partial y + f_z \partial z $\\
\noindent{\color{smalt(darkpowderblue)}\rule{\linewidth}{.2mm}}
%----------------------------------------------------
\begin{problem}
1-43(odd),44,48,49,50,51,53,59,61,65,71,72(a,d),73,75,77,87*,89*,93,94
\end{problem}
\section{Directional derivatives and Gradient vector}
$f_x=\lim_{h\to0} \cfrac{f(x+h,y)-f(x,y) }{h}$\\
$f_y = \lim_{h \to 0} \cfrac{f(x,y+h) - f(x,y)}{h}$\\
\noindent{\color{smalt(darkpowderblue)}\rule{\linewidth}{.2mm}}

\begin{definition}
the direction derivative of f at $(x_\circ , y_\circ)$ in the direction of the limit vector ${\overrightarrow{u}} = <a,b> is D_{\overrightarrow{u}} f(x_\circ,y_\circ)=\lim\cfrac{f(x+a h,y+b h) - f(x,y)}{h}$
\end{definition} 

Gradient vector:\\
\begin{definition}
if $z = f(x,y) , then the gradient of f at (x_\circ,y_\circ) is \nabla f = < f_x (x_\circ ,y_\circ),f_y(x_\circ , y_\circ)>$
\end{definition}
%------------------------------------------------------
\noindent{\color{smalt(darkpowderblue)}\rule{\linewidth}{.2mm}}
\begin{example}
if $ f(x,y) = x^{2} - y^{2} find \nabla f\mid_(1,2)$\\
$f_x = 2x \rightarrow f_x(1,2) = 2 $\\
$f_y = 2y \rightarrow f_y(1,2) = -4$\\
{\color{smalt(darkpowderblue)}\underline{Solution:}}\\,$\nabla f\mid_(1,2) = <2,-4>$
\end{example}

%----------------------------------------------------
\begin{theorem}
 $D_{\overrightarrow{u}} f(x_\circ,y_\circ) = \nabla f .{\overrightarrow{u}}$
\end{theorem}
%-----------------------------------------------------
\begin{example}
if $f(x,y) = sin x + e ^{xy}$ find the directional of f at (0,1) in the direction of ${\overrightarrow{u}} = <3,-4>$\\
so, $D_{\overrightarrow{u}} f(0,1)= \nabla f .{\overrightarrow{u}}$\\
$=<2,0> . <\cfrac{3}{5},\cfrac{4}{5}>\\
=\cfrac{6}{10} + 0\cfrac{4}{5} = (0,6)$\\
$\nabla f = <f_x,f_y> . <2,0>\\
f_x = cos x + y e^{xy} \\
f_y = x e^{xy} \\
f_x(0,1) = 2\\
f_y(0,1) = 0 $\\
function of three variables $w = f(x,y,z)$\\
$\nabla f = ,f_x,f_y,f_z>$\\
$D_{\overrightarrow{u}} f(a,b,c) = \nabla f . {\overrightarrow{u}}$\\
\end{example}
\noindent{\color{smalt(darkpowderblue)}\rule{\linewidth}{.2mm}}
%----------------------------------------------------------------------
\begin{example}
find the direction derivative of $f(x,y,z) = \cfrac{x - y}{z} + x^{2} + e^{y}$\\
at $p(1,0,1) in the directional to the point Q (-1,2,0)$\\
{\color{smalt(darkpowderblue)}\underline{Solution:}}\\ $D_{\overrightarrow{u}} f(-1,2,0) = \nabla f . {\overrightarrow{u}} = <3,0,-1>.<\cfrac{2}{3},\cfrac{2}{3},\cfrac{-1}{3}> = -2 + 0 + \cfrac{1}{3} = -1\cfrac{2}{3}$\\
${\overrightarrow{u}} = \cfrac{{\overrightarrow{PQ}}}{|{\overrightarrow{PQ}}|} = <\cfrac{-2}{3} , \cfrac{2}{3} , \cfrac{-1}{0}>$\\
$f_x = \cfrac{1}{z} + 2x , f_x(1,0,1) = 1+2 = 3$\\
$f_y = \cfrac{-1}{z} + e^{y} \rightarrow f_y(91,0,1) = -1 +1 = 0 $\\
$f_y = \cfrac{y-x}{z^{2}} , f_y(1,0,1) = \cfrac{0-1}{1^{2}} = -1 $\\
$\nabla f = <3,0,-1>$\\
\end{example}
\noindent{\color{smalt(darkpowderblue)}\rule{\linewidth}{.2mm}}
%------------------------------------------------------
\begin{example}
if $D_{\overrightarrow{u}} f = 3    \nabla 2f .{\overrightarrow{u}}$\\
if $D_{\overrightarrow{u}} 2f = 6$\\
$D_{\overrightarrow{u}} -2f = -6$\\
$D_{\overrightarrow{2u}} f = 3$\\
$D_{\overrightarrow{-2u}} f = -3 $
\end{example}
\noindent{\color{smalt(darkpowderblue)}\rule{\linewidth}{.2mm}}
Question $z = f(x,y)(x_\circ , y_\circ)$\\
${\overrightarrow{u}} = ??$\\
find ${\overrightarrow{u}}$ that maximize $D_{\overrightarrow{u}} f(x_\circ , y_\circ )= \nabla f .{\overrightarrow{u}} = |\nabla f| |{\overrightarrow{u}}| cos$\\


\begin{theorem}
 \begin{enumerate}
    \item the max directional derivative of f at $(x_\circ , y_\circ) is |\nabla f|$ and it accrues if${\overrightarrow{u}}$ has the same direction of $\nabla f$ .
    \item the min directional derivative of f is $-|\nabla f | $ and it accrues if${\overrightarrow{u}}$ has the opposite direction of $\nabla f$.
\end{enumerate}
\end{theorem}

\begin{example}
Let $z = f(x.y) = xe^{y}$ find the max directional derivative at (2,0).\\
{\color{smalt(darkpowderblue)}\underline{Solution:}}\\
$\nabla f = <f_x ,f_y> , f_x = e^{y} f_x(2,0) = 1 , f_y = xe^{y} f_y (2,0) = 2$\\
$\nabla f = <1,2>$\\
$max D_{\overrightarrow{u}} f(2,0) = |\nabla f| = \surd 5$\\
it occurs if ${\overrightarrow{u}}$ has the same direction of $<1,2>$\\
max directional derivative $\leftrightarrow$ max rate of change \\
                           $\leftrightarrow$ increasing most rapidly \\
min directional derivative $\leftrightarrow$ min rate of change\\
                           $\leftrightarrow$ decreasing most rapidly

\end{example} 
\noindent{\color{smalt(darkpowderblue)}\rule{\linewidth}{.2mm}}
%===================================================
\underline{\color{smalt(darkpowderblue)}Tangent plane for level surfaces}\\
\noindent\begin{minipage}{0.5\textwidth}
$k = f(x,y,z)$ level surfaces\\
to find the plane we need 
    1.point $(x_\circ , y_\circ , z_\circ)$
    2.${\overrightarrow{n}} = \nabla f$\\
\end{minipage}
\noindent\begin{minipage}{0.5\textwidth}
\begin{center}
   \includegraphics[width=5cm]{rr1.png}\\
\end{center}\end{minipage}
\noindent{\color{smalt(darkpowderblue)}\rule{\linewidth}{.2mm}}

\begin{example}
\begin{enumerate}
    \item find the equation of the tangent plane to $\cfrac{x^{2}}{4} + y^{2} + \cfrac{z^{2}}{9} = 3 at (2,1,3)$
    \item find the equation of the normal liner
\end{enumerate}

{\color{smalt(darkpowderblue)}\underline{Solution:}}
\begin{enumerate}
    \item plane ! point (2,1.3) \\
${\overrightarrow{n}} = \nabla f = <\cfrac{2x}{4},2y,\cfrac{2}{9} = <1,2,\cfrac{2}{3}$ \\
$|(x-2) + 2(y-1) + \cfrac{2}{3} (z-3)|$
\item point (2,1,3)\\
${\overrightarrow{n}} = <1,2,\cfrac{2}{3}> , x=2+t \\
                                          , y=1+2t\\
                                          , z=3+\cfrac{2}{3}t $\\
$\nabla f = <f_x,f_y,f_z>$\\
$D_{\overrightarrow{u}} f(x_\circ , y_\circ , z_\circ) = \nabla f .{\overrightarrow{u}}$\\
max $D_{\overrightarrow{u}} f = |\nabla f|$ it accrues if $\nabla f ,{\overrightarrow{u}}$ have the same direction \\
F(x,y,z) = K the n the normal to the tangent $D_{\overrightarrow{n}} f = |\nabla f|$ 
\end{enumerate}
\end{example}